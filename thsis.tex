\documentclass[a4paper,11pt,oneside,openany]{jsbook}
%
\usepackage[dvipdfmx]{graphicx}
\usepackage{amsmath,amssymb}
\usepackage{bm}
\usepackage{graphicx}
\usepackage{ascmac}
%
\setlength{\textwidth}{\fullwidth}
\setlength{\textheight}{40\baselineskip}
\addtolength{\textheight}{\topskip}
\setlength{\voffset}{-0.55in}
%
\title{AirbridgeのNbとAlの接触評価}
\author{大内 崇}
\date{\today}

\begin{document}
\begin{center}
  \huge 平成31年度卒業論文\par
  \vspace{15mm}
  \huge AirbridgeのNbとAlの接触評価 \par
  \vspace{15mm}
  \LARGE 大内 崇 \par
  \vspace{100mm}
  \Large \today \par
  \vspace{15mm}
  \Large 蔡研究室 \par
  \vspace{10mm}
  \Large 学籍番号 2216023  指導教員(主査) 蔡 兆申 教授\par
  \vspace{10mm}
\end{center}
\thispagestyle{empty}
\clearpage
\addtocounter{page}{-1}
\newpage
\setcounter{tocdepth}{3}
%
\tableofcontents
%
\chapter{序章}
\section{背景}
 量子コンピュータは
\section{目的}
 Airbrigeは

\chapter{原理}
\section{CPW(Coplanar waveguide)}
\section{Airbrige}

\chapter{資料作成}
\section{CPWの作成手順}
\section{Airbrigeの作成手順}

\chapter{測定環境・手順}
\section{測定機器}
\section{手順}

\chapter{実験結果}
\section{NbとAlのCPW}
\section{Nbのみ}

\chapter{考察}


\chapter{結論}
\section*{謝辞}
\addcontentsline{toc}{chapter}{謝辞}
%
\begin{thebibliography}{99}
  \bibitem{1}
\end{thebibliography}
%
% END DOCUMENT
\end{document}

