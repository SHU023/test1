量子情報計科学技術は、量子計算、量子通信などの新しい性能を有する次世代の科学技術として期待されている。量子計算は、
現在のコンピュータ(古典コンピュータ)では処理できない、重要な問題を効率的に解くことができる、新しい計算パラダイムである。
これまでの情報処理技術の進歩は、情報処理のためのエネルギーの限界、ムーアの法則の限界、古典計算機の情報処理能力の原理的限界
などの多くの限界をじきに迎えることになる。量子情報技術はこれらの限界にとらわれず、計算速度の指数関数的な増加や、計算に必要
なエネルギーの桁違いな削減が可能となる注目すべき科学技術である。

超電導量子回路を用いた量子ビットの歴史は、1999年NECのグループが世界に先駆けて発表してから今日まで目覚ましい発展を遂げている。
