量子情報計科学技術は、量子計算、量子通信などの新しい性能を有する次世代の科学技術として期待されている。量子計算は、
現在のコンピュータでは処理できない、重要な問題を効率的に解くことができる、新しい計算パラダイムである。
これまでの情報処理技術の進歩は、情報処理のためのエネルギーの限界、ムーアの法則の限界、古典計算機の情報処理能力の原理的限界
などの多くの限界をじきに迎えることになる。一方、量子コンピュータはこれらの限界にとらわれず、計算速度の指数関数的な増加や、
計算に必要なエネルギーの桁違いな削減が可能となる注目すべき科学技術である。

量子コンピュータは重ね合わせと干渉効果を利用することで、高速に解くことができると理論的に示されており、その基本的構成要素は
量子ビットである。超電導量子回路を用いた量子ビットは集積可能性と設計自由度が非常に高く、1999年NECのグループが世界に先駆け
て発表してから今日まで目覚ましい発展を遂げている。この超伝導量子ビットの情報の読み出しや操作に使う共振器としてCPW(coplanar waveguide)がある。
CPWは誘電体基板表面に中央のstrip導体とその両サイドのground planeから構成され、他の表面波線路と比べ作製やshuntが容易で、
損失率が低い特徴がある。特性インピーダンスは中央のstripの幅とground planeまでの幅の割合で決まり、柔軟に調整可能である。
CPWには準TEMの伝搬以外にもslotline modesとよばれるmodeが励起される問題がる。slotline modesはgroundの効きが良くない場合に
非対称性や不連続性によって励起される。そのためground plane間にAirbrigeと呼ばれる橋をかけることによってslotline modesを抑制
できる。